\documentclass[10pt,oneside]{article}

\usepackage[T1]{fontenc}
\usepackage{fontawesome}

\usepackage[paper=a4paper,margin=2cm,bottom=2.5cm]{geometry}
\usepackage[sfdefault,light,condensed]{roboto}
\usepackage[export]{adjustbox}
\usepackage[usenames,dvipsnames,table]{xcolor}

\usepackage{amsmath,amssymb,array,fancyhdr,graphicx,enumitem,lastpage,multicol,tabularx,textcomp,titlesec}
\usepackage{mathtools}

\setlength\extrarowheight{1pt}
\setlength\parindent{0cm}
\renewcommand\headrule{}
\setlength{\footskip}{1.25cm}

\pagestyle{fancy}

\definecolor{BoxHeaderBG}{RGB}{50, 50, 50}
\definecolor{BoxHeaderText}{RGB}{255, 255, 255}

\newcommand{\BoxHeader}[2]{
    \multicolumn{#1}{| >{\bfseries\footnotesize\cellcolor{BoxHeaderBG}\arraybackslash}l |}{
        \textcolor{BoxHeaderText}{#2}
    }
}

\definecolor{ATLHeaderBG}{RGB}{65, 190, 30}
\definecolor{ATLHeaderText}{RGB}{0, 0, 0}

\definecolor{ATLSkillBG}{RGB}{215, 230, 210}
\definecolor{ATLSkillText}{RGB}{0, 0, 0}

\definecolor{DefinitionBoxHeaderBG}{RGB}{30, 30, 110}
\definecolor{DefinitionBoxHeaderText}{RGB}{255, 255, 255}

\definecolor{FormativeHeaderBG}{RGB}{150, 30, 150}
\definecolor{FormativeHeaderText}{RGB}{255, 255, 255}

\definecolor{GlobalContextHeaderBG}{RGB}{255, 255, 150}
\definecolor{GlobalContextHeaderText}{RGB}{0, 0, 0}

\definecolor{KeyConceptHeaderBG}{RGB}{15, 225, 225}
\definecolor{KeyConceptHeaderText}{RGB}{0, 0, 0}

\definecolor{RelatedConceptHeaderBG}{RGB}{15, 170, 170}
\definecolor{RelatedConceptHeaderText}{RGB}{0, 0, 0}

\definecolor{QuestionHeaderBG}{RGB}{240, 240, 240}
\definecolor{QuestionHeaderText}{RGB}{0, 0, 0}

\definecolor{SolutionHeaderBG}{RGB}{225, 150, 110}
\definecolor{SolutionHeaderText}{RGB}{0, 0 , 0}

\definecolor{SummativeHeaderBG}{RGB}{195, 15, 15}
\definecolor{SummativeHeaderText}{RGB}{255, 255, 255}

\definecolor{WarningHeaderBG}{RGB}{250, 220, 125}
\definecolor{WarningHeaderText}{RGB}{0, 0, 0}

\newcommand{\ATLHeader}[1]{
    \cellcolor{ATLHeaderBG}\textcolor{ATLHeaderText}{
        \bfseries\footnotesize
        ATL SKILL (#1) \hfill \faGears
    }
}

\newcommand{\ATLSkill}[1]{
    \cellcolor{ATLSkillBG}\textcolor{ATLSkillText}{
        \itshape #1
    }
}

\newcommand{\DefinitionBoxHeader}{
    \cellcolor{DefinitionBoxHeaderBG}\textcolor{DefinitionBoxHeaderText}{
        \bfseries\footnotesize
        DEFINITIONS \hfill \faPencil
    }
}

\newcommand{\FormativeHeader}{
    \cellcolor{FormativeHeaderBG}\textcolor{FormativeHeaderText}{
        \bfseries\footnotesize
        FORMATIVE ASSESSMENT \hfill \faComments
    }
}

\newcommand{\GlobalContextHeader}[1]{
    \cellcolor{GlobalContextHeaderBG}\textcolor{GlobalContextHeaderText}{
        \bfseries\footnotesize
        GLOBAL CONTEXT (#1) \hfill \faGlobe
    }
}

\newcommand{\KeyConceptHeader}[1]{
    \cellcolor{KeyConceptHeaderBG}\textcolor{KeyConceptHeaderText}{
        \bfseries\footnotesize
        KEY CONCEPT (#1) \hfill \faKey
    }
}

\newcommand{\RelatedConceptHeader}[1]{
    \cellcolor{RelatedConceptHeaderBG}\textcolor{RelatedConceptHeaderText}{
        \bfseries\footnotesize
        RELATED CONCEPT (#1) \hfill \faLink
    }
}

\newcommand{\SolutionHeader}[1]{
    \cellcolor{SolutionHeaderBG}\textcolor{SolutionHeaderText}{
        \bfseries\footnotesize 
        #1 \hfill \faPaste
    }
}

\newcommand{\SummativeHeader}{
    \cellcolor{SummativeHeaderBG}\textcolor{SummativeHeaderText}{
        \bfseries\footnotesize
        SUMMATIVE ASSESSMENT \hfill \faCheck
    }
}

\newcommand{\WarningHeader}[1]{
    \cellcolor{WarningHeaderBG}\textcolor{WarningHeaderText}{
        \textbf{\footnotesize\faExclamationTriangle} {\normalfont #1} \hfill \textbf{\footnotesize\faExclamationTriangle}
    }
}
\newcounter{QuestionCounter}

\newcommand{\QuestionBox}[1]{
    \stepcounter{QuestionCounter}
    \cellcolor{QuestionHeaderBG}\textcolor{QuestionHeaderText}{
        {\bfseries\scriptsize Q\theQuestionCounter} #1
    }
}

\newcommand{\boxwidth}{\linewidth}

\definecolor{ATLHeaderBG}{RGB}{65, 190, 30}
\definecolor{ATLHeaderText}{RGB}{0, 0, 0}

\definecolor{ATLSkillBG}{RGB}{215, 230, 210}
\definecolor{ATLSkillText}{RGB}{0, 0, 0}

\definecolor{DefinitionBoxHeaderBG}{RGB}{30, 30, 110}
\definecolor{DefinitionBoxHeaderText}{RGB}{255, 255, 255}

\definecolor{FormativeHeaderBG}{RGB}{150, 30, 150}
\definecolor{FormativeHeaderText}{RGB}{255, 255, 255}

\definecolor{GlobalContextHeaderBG}{RGB}{255, 255, 150}
\definecolor{GlobalContextHeaderText}{RGB}{0, 0, 0}

\definecolor{KeyConceptHeaderBG}{RGB}{15, 225, 225}
\definecolor{KeyConceptHeaderText}{RGB}{0, 0, 0}

\definecolor{RelatedConceptHeaderBG}{RGB}{15, 170, 170}
\definecolor{RelatedConceptHeaderText}{RGB}{0, 0, 0}

\definecolor{QuestionHeaderBG}{RGB}{240, 240, 240}
\definecolor{QuestionHeaderText}{RGB}{0, 0, 0}

\definecolor{SolutionHeaderBG}{RGB}{225, 150, 110}
\definecolor{SolutionHeaderText}{RGB}{0, 0 , 0}

\definecolor{SummativeHeaderBG}{RGB}{195, 15, 15}
\definecolor{SummativeHeaderText}{RGB}{255, 255, 255}

\newcommand{\ATLHeader}[1]{
    \cellcolor{ATLHeaderBG}\textcolor{ATLHeaderText}{
        \bfseries\footnotesize
        ATL SKILL (#1) \hfill \faGears
    }
}

\newcommand{\ATLSkill}[1]{
    \cellcolor{ATLSkillBG}\textcolor{ATLSkillText}{
        \itshape #1
    }
}

\newcommand{\DefinitionBoxHeader}{
    \cellcolor{DefinitionBoxHeaderBG}\textcolor{DefinitionBoxHeaderText}{
        \bfseries\footnotesize
        DEFINITIONS \hfill \faPencil
    }
}

\newcommand{\FormativeHeader}{
    \cellcolor{FormativeHeaderBG}\textcolor{FormativeHeaderText}{
        \bfseries\footnotesize
        FORMATIVE ASSESSMENT \hfill \faComments
    }
}

\newcommand{\GlobalContextHeader}[1]{
    \cellcolor{GlobalContextHeaderBG}\textcolor{GlobalContextHeaderText}{
        \bfseries\footnotesize
        GLOBAL CONTEXT (#1) \hfill \faGlobe
    }
}

\newcommand{\KeyConceptHeader}[1]{
    \cellcolor{KeyConceptHeaderBG}\textcolor{KeyConceptHeaderText}{
        \bfseries\footnotesize
        KEY CONCEPT (#1) \hfill \faKey
    }
}

\newcommand{\RelatedConceptHeader}[1]{
    \cellcolor{RelatedConceptHeaderBG}\textcolor{RelatedConceptHeaderText}{
        \bfseries\footnotesize
        RELATED CONCEPT (#1) \hfill \faLink
    }
}

\newcommand{\SolutionHeader}[1]{
    \cellcolor{SolutionHeaderBG}\textcolor{SolutionHeaderText}{
        \bfseries\footnotesize 
        #1 \hfill \faPaste
    }
}

\newcommand{\SummativeHeader}{
    \cellcolor{SummativeHeaderBG}\textcolor{SummativeHeaderText}{
        \bfseries\footnotesize
        SUMMATIVE ASSESSMENT \hfill \faCheck
    }
}

\newcommand{\QuestionBox}[1]{
    \cellcolor{QuestionHeaderBG}\textcolor{QuestionHeaderText}{
        #1
    }
}

\lhead{\footnotesize\texttt{U\UnitNumber: \UnitTitle \\ L\LessonNumber: \LessonTitle}}
\rhead{\footnotesize\ttfamily [DESIGN/\CourseName/U\UnitNumber/L\LessonNumber]\\\ }

\lfoot{\includegraphics[height=2cm,valign=c]{Files/logo}}
\cfoot{\footnotesize DESIGN/\CourseName/U\UnitNumber/L\LessonNumber\ | \LessonTitle \\ Woodstock School | Mussoorie, Uttarakhand, India}
\rfoot{\includegraphics[height=2cm,valign=c]{Files/ib-world-school-logo-1-colour}}

\titleformat{\section}{\normalfont\Large\bfseries}{}{0em}{}[{\titlerule[0.5pt]}]
\titleformat{\subsection}{\normalfont\large\bfseries}{}{0em}{}


\usepackage{circuitikz}
\usetikzlibrary{arrows,calc,shapes.misc}

\tikzset{cross/.style={cross out, draw, 
         minimum size=2*(#1-\pgflinewidth), 
         inner sep=0pt, outer sep=0pt}}

\def\CourseName{MYP3}

\def\LessonNumber{03}
\def\LessonTitle{Series \& Parallel Circuits}

\def\UnitNumber{01}
\def\UnitTitle{Circuits \& Electronics}

\begin{document}
    % Core Components (Remove when no longer required!)
    \thispagestyle{empty}
    \begin{tabularx}{\boxwidth}{| X | }
        \hline
        \ATLHeader{} \\\hline
        \ATLSkill{} \\\hline
        \QuestionBox{} \\\hline
    \end{tabularx}

    \begin{tabularx}{\boxwidth}{| X |}
        \hline
        \DefinitionBoxHeader \\\hline
    \end{tabularx}

    \begin{tabularx}{\boxwidth}{| X |}
        \hline
        \GlobalContextHeader{}\\\hline
    \end{tabularx}

    \begin{tabularx}{\boxwidth}{| X |}
        \hline
        \KeyConceptHeader{} \\\hline
    \end{tabularx}

    \begin{tabularx}{\boxwidth}{| X |}
        \hline
        \RelatedConceptHeader{} \\\hline
    \end{tabularx}

    \begin{tabularx}{\boxwidth}{| X |}
        \hline
        \SolutionHeader{SOLUTION} \\\hline
    \end{tabularx}

    \begin{tabularx}{\boxwidth}{| X | }
        \hline
        \FormativeHeader \\\hline
        \QuestionBox{} \\\hline
    \end{tabularx}

    \begin{tabularx}{\boxwidth}{| X |}
        \hline
        \SummativeHeader \\\hline
        \QuestionBox{}\\\hline
    \end{tabularx}

    \newpage

    \begin{center}
        \huge\bfseries
        \LessonTitle
    \end{center}

    \section{Before You Begin}

    \pagebreak

    \section{Technical Background}
    So far, all of the circuits we have built have revolved around a single ``loop'' of electric current. These simple circuits are called \emph{series circuits}.

    \medskip
    This lesson focuses on the difference between building a circuit in \emph{series} and \emph{parallel}, as defined below.

    \subsection{Series Circuits}

    \begin{minipage}{0.26\boxwidth}
        \centering
        \begin{circuitikz}
            \draw (0, 1) node[left,yshift=-5] {+} to [battery1] (0, 0);
            \node at (-0.75, 0.5) {3V};
            \draw (0, 1) |- (1, 2) to [full led] (2, 2) -- (3, 2) |- (2, -1) to [full led] (1, -1) -| (0, 0);
        \end{circuitikz}
    \end{minipage}
    \begin{minipage}{0.025\boxwidth}
        \ 

    \end{minipage}
    \begin{minipage}{0.35\boxwidth}
        In a simple series circuit such as the one given here, electrical energy passes through one component before entering the next.
    \end{minipage}
    \begin{minipage}{0.26\boxwidth}
        \centering
        \begin{circuitikz}
            \draw (0, 1) node[left,yshift=-5] {+} to [battery1] (0, 0);
            \node at (-0.75, 0.5) {3V};
            \draw (0, 1) |- (1, 2) to [full led] (2, 2) -- (3, 2) |- (2, -1) to [full led] (1, -1) -| (0, 0);

            \draw[->, >=triangle 45,very thick, red] (0.5, 0.75) |- (2.5, 1.5) |- (2.5, -0.5) -| (0.5, 0.5);
        \end{circuitikz}
    \end{minipage}

    \subsection{Parallel Circuits}
    \begin{minipage}{0.3\boxwidth}
        \centering
        \begin{circuitikz}
            \draw (0, 1) node[left,yshift=-5] {+} to [battery1] (0, 0);
            \node at (-0.75, 0.5) {3V};
            \draw (0, 1) |- (2, 2) -| (3, 1) to [full led] (3, 0);
            \draw (1.5, 2) -- (1.5, 1) to [full led] (1.5, 0);
            \draw (3, 0) -- (3, -1) -| (0, 0);
            \draw (1.5, 0) -- (1.5, -1);
        \end{circuitikz}
    \end{minipage}
    \begin{minipage}{0.34\boxwidth}
        In parallel circuits, electrical energy splits along multiple paths. This allows voltage to remain the same across all components, while dividing current.
    \end{minipage}
    \begin{minipage}{0.3\boxwidth}
        \centering
        \begin{circuitikz}
            \draw (0, 1) node[left,yshift=-5] {+} to [battery1] (0, 0);
            \node at (-0.75, 0.5) {3V};
            \draw (0, 1) |- (2, 2) -| (3, 1) to [full led] (3, 0);
            \draw (1.5, 2) -- (1.5, 1) to [full led] (1.5, 0);
            \draw (3, 0) -- (3, -1) -| (0, 0);
            \draw (1.5, 0) -- (1.5, -1);

            \draw[->, >=triangle 45,very thick, red] (0.5, 0.75) |- (1.4, 1.5) |- (1.25, -0.5) -- (0.5, -0.5) -- (0.5, 0.5);
            \draw[->, >=triangle 45,very thick, red] (0.5, 0.75) |- (2.5, 1.5) |- (2.5, -0.5) -- (0.5, -0.5) -- (0.5, 0.5);

            \draw[->,>=stealth,very thick, blue] (0.75, 1.4) -- (1.95, 1.4);
            \draw[->,>=stealth,very thick, blue] (1.05, 1.15) +(100:0.25) arc 
            (100:-20:0.25);
        \end{circuitikz}
    \end{minipage}

    \subsection{Resistors in Series}
    Because resistors are used to restrict the flow of electricity, or \emph{current}, they behave differenty in series circuits versus parallel circuits.

    \medskip
    In series circuits, the \emph{total resistance} in the circuit is calculated as the sum of each resistor.
    
    \subsubsection*{Example}
    \begin{minipage}{0.3\boxwidth}
        \begin{center}
            \begin{circuitikz}
                \draw (0, 1) node[left,yshift=-5] {+} to [battery1] (0, 0);
                \node at (-0.75, 0.5) {3V};
                \draw (0, 1) |- (1, 2) to [european resistor,label=$220\ \Omega$] (2, 2) -| (3, 1) to [european resistor,label=$1\ \text{k}\Omega$] (3, 0) |- (2, -1) to [european resistor,label=$330\ \Omega$] (1, -1) -| (0, 0);
            \end{circuitikz}
        \end{center}
    \end{minipage}
    \begin{minipage}{0.375\boxwidth}
        \[\begin{aligned}
            \text{R}_\text{total} &= \text{R}_1 + \text{R}_2 + \text{R}_3 \\
            &= 220 + 1000 + 330\\
            &= 1550\ \Omega
        \end{aligned}\]
    \end{minipage}

    \bigskip
    \begin{tabularx}{\boxwidth}{| X |}
        \hline
        \SolutionHeader{Calculating Total Resistance in a Series Circuit} \\\hline
        The above example yields the following formula:\\
        \hfill R$_{\text{total}} = \text{R}_{1} + \text{R}_{2} + \cdots + \text{R}_{n}$ \hfill\, \\
        This is a very simple formula, but can be applied in a variety of ways to create circuits with various target resistances.\\\hline
    \end{tabularx}

    \pagebreak
    \subsection{Resistors in Parallel}
    Because parallel circuits \emph{divide current}, it becomes a bit harder to determine total resistance in the type of circuits seen below.

    \subsubsection*{Example \#1}
    Rarely, all resistors in the circuit will have the same value as in this first example. In that case, all you need to do is divide the resistor value by the number of parallel paths in the circuit.

    \bigskip
    \begin{minipage}{0.45\boxwidth}
        \begin{circuitikz}
            \draw (0, 1) node[left,yshift=-5] {+} to [battery1] (0, 0);
            \node at (-0.75, 0.5) {3V};
            \draw (0, 1) |- (1.75, 2) -- (1.75, 1) to [european resistor,label=\small$330\ \Omega$] (1.75, 0) -- (1.75, -1) -| (0, 0);
            \draw (1, 2) -| (3.5, 1) to [european resistor,label=\small$330\ \Omega$] (3.5, 0) -- (3.5, -1) -| (0, 0);
            \draw (3.5, 2) -| (5.25, 1) to [european resistor,label=\small$330\ \Omega$] (5.25, 0) -- (5.25, -1) -| (0, 0);
        \end{circuitikz}
    \end{minipage}
    \begin{minipage}{0.2\boxwidth}
        \[\begin{aligned}
            \text{R}_{\text{total}} &= \frac{\text{R}}{3} \\[8pt]
            &= \dfrac{330}{3} \\[8pt]
            &= 110\ \Omega
        \end{aligned}\]
    \end{minipage}

    \subsubsection*{Example \#2}
    The result is less obvious for more complex parallel circuits. To understand what is going on, let's make a small chart of values for the current passing through each resistor. Remember: in a parallel circuit such as this one the voltage across all components remains the same. Keep in mind that \emph{Ohm's Law} gives us a formula for current: $I = \frac{V}{R}$.

    \bigskip
    \begin{minipage}{0.475\boxwidth}
        \begin{circuitikz}
            \draw (0, 1) node[left,yshift=-5] {+} to [battery1] (0, 0);
            \node at (-0.75, 0.5) {3V};
            \draw (0, 1) |- (1.75, 2) -- (1.75, 1) to [european resistor,label=\small$220\ \Omega$] (1.75, 0) -- (1.75, -1) -| (0, 0);
            \draw (1, 2) -| (3.5, 1) to [european resistor,label=\small$470\ \Omega$] (3.5, 0) -- (3.5, -1) -| (0, 0);
            \draw (3.5, 2) -| (5.25, 1) to [european resistor,label=\small$1\ \text{k}\Omega$] (5.25, 0) -- (5.25, -1) -| (0, 0);
        \end{circuitikz}
    \end{minipage}
    \begin{minipage}{0.5\boxwidth}
        \begin{center}
            \newcolumntype{C}{>{\centering\arraybackslash}X}
            \renewcommand\arraystretch{1.5}
            \begin{tabularx}{\boxwidth}{| c | C | C | C |}
                \hline
                \cellcolor{black!25} & \textbf{R$_1$} & \textbf{R$_2$} & \textbf{R$_3$}\\\hline
                \textbf{V} & 3 & 3 & 3\\\hline
                \textbf{R} & 220 & 470 & 1000\\\hline    
                \textbf{I} & $\frac{3}{220} \approx 0.0136$ & $\frac{3}{470} \approx 0.0064$ & $\frac{3}{1000} = 0.0030$ \\\hline
            \end{tabularx}
        \end{center}
    \end{minipage}

    \bigskip
    From the above table, we can see that the \emph{total current} in the circuit is approximately $0.023$ amps (23 mA). Once again using \emph{Ohm's Law} $\left(R = \frac{V}{I}\right)$ gives us a \emph{total resistance} of: $R = \frac{3}{0.023} \approx 130.435\ \Omega$.

    \bigskip
    \renewcommand\arraystretch{1}
    \begin{tabularx}{\boxwidth}{| X |}
        \hline
        \SolutionHeader{Calculating Total Resistance of a Parallel Circuit}\\\hline
        The above work can be summarized by the following formula:\\
        \hfill R$_{\text{total}} = \dfrac{1}{\dfrac{1}{\text{R}_1} + \dfrac{1}{\text{R}_2} + \cdots + \dfrac{1}{\text{R}_n}}$ \hfill\, \\
        Although it looks complicated, the end result of this formula is identical to making a chart of values similar to what we did previously.\\\hline
    \end{tabularx}
   
    \bigskip
    \begin{tabularx}{\boxwidth}{| X | }
        \hline
        \ATLHeader{Communication Skills} \\\hline
        \ATLSkill{...use and interpret a range of discipline-specific terms and symbols...} \\\hline
        \QuestionBox{Apply the formula above to calculate the total resistance of a parallel circuit using the given resistor values ($220\ \Omega$, $470\ \Omega$, and $1000\ \Omega$).} \\\hline
        \ \\[1.75cm]\hline
        \ATLSkill{...make inferences and draw conclusions...}\\\hline
        \QuestionBox{Explain why using the formula did \emph{not} yield the exact same value as in our example.}\\\hline
        \ \\[1.75cm]\hline
    \end{tabularx}
   
    \pagebreak

    \section{Developing Technical Skills}

    \subsubsection*{Circuit \#7: Building a Parallel Circuit}
    % parallel circuit two LEDs
    \begin{center}
        \begin{tikzpicture}
            \draw[line width=6mm,YellowOrange] (0, 0) -- (0, 5) -- (10, 5) -- (10, 1);
            \draw[line width=6mm,YellowOrange] (10, -1) -- (10, -5) -- (0, -5) -- (0, 0);

            \draw[line width=6mm,YellowOrange] (5, 5) -- (5, 1);
            \draw[line width=6mm,YellowOrange] (5, -1) -- (5, -5);

            \fill[black!25] (0, 0) circle (10mm);            
            \draw[line width=6mm, dashed,YellowOrange!50,draw opacity=0.5] (0, -0.25) -- (0, -1);
            \draw[line width=6mm,YellowOrange] (0, 0.25) -- (0, 1);
            \node[align=center] at (0, 0) {CR2032};

            \draw[very thick,black!50] (5, 2.25) -- (5, -1.75);
            \fill[left color=red, right color=black!70] ([shift=(-60:2.5mm)]5,0) arc (-60:240:2.5mm);

            \draw[very thick,black!50] (10, 2.25) -- (10, -1.75);
            \fill[left color=red, right color=black!70] ([shift=(-60:2.5mm)]10,0) arc (-60:240:2.5mm);

            \draw[->,>=triangle 45,very thick,red] (1.4, 2) node[right] {A} -- (0.4, 2);
            \draw[->,>=triangle 45,very thick,black] (1.4, -2) node[right] {A} -- (0.4, -2);

            \draw[->,>=triangle 45,very thick,red] (3.6, 3.5) node[left] {B} --(4.6, 3.5);
            \draw[->,>=triangle 45,very thick,black] (3.6, -3.5) node[left] {B} -- (4.6, -3.5);

            \draw[->,>=triangle 45,very thick,red] (8.6, 3.5) node[left] {C} --(9.6, 3.5);
            \draw[->,>=triangle 45,very thick,black] (8.6, -3.5) node[left] {C} -- (9.6, -3.5);            
        \end{tikzpicture}
    \end{center}

    % use of multimeter
    \subsubsection*{Using a Multimeter}

    % series resistors + multimeter
    \subsubsection*{Circuit \#8: Resistors in Series}
    \begin{center}
        \begin{tikzpicture}
            \draw[line width=6mm,YellowOrange] (0, 0) |- (4, 5);
            \draw[line width=6mm,YellowOrange] (6, 5) -| (10, 1);
            \draw[line width=6mm,YellowOrange] (10, -1) |- (6, -5);
            \draw[line width=6mm,YellowOrange] (4, -5) -- (0, -5) -- (0, 0);

            \fill[black!25] (0, 0) circle (10mm);            
            \draw[line width=6mm, dashed,YellowOrange!50,draw opacity=0.5] (0, -0.25) -- (0, -1);
            \draw[line width=6mm,YellowOrange] (0, 0.25) -- (0, 1);
            \node[align=center] at (0, 0) {CR2032};
            
            \draw[->,>=triangle 45,very thick,red] (1.4, 2) node[right] {A} -- (0.4, 2);
            \draw[->,>=triangle 45,very thick,black] (1.4, -2) node[right] {A} -- (0.4, -2);
            
            \draw[->,>=triangle 45,very thick,red] (2, 3.6) node[below] {B, C} -- (2, 4.6);
            \draw[->,>=triangle 45,very thick,black] (8, 3.6) node[below] {B} -- (8, 4.6);

            \draw[->,>=triangle 45,very thick,black] (8.6, -3.5) node[left] {C} -- (9.6, -3.5);

            \coordinate (R1) at (10, -2);
            %220 Ohms
            \draw[very thick, black!50] (R1) -- ($(R1) + (0, 4)$);
            \draw[fill=Tan] ($(R1) + (0.25, 1.4)$) rectangle ($(R1) + (-0.25, 2.6)$);
            \draw[fill=Red] ($(R1) + (0.25, 2.4)$) rectangle ($(R1) + (-0.25, 2.3)$);
            \draw[fill=Red] ($(R1) + (0.25, 2.2)$) rectangle ($(R1) + (-0.25, 2.1)$);
            \draw[fill=Brown] ($(R1) + (0.25, 2.0)$) rectangle ($(R1) + (-0.25, 1.9)$);
            \draw[fill=Goldenrod] ($(R1) + (0.25, 1.6)$) rectangle ($(R1) + (-0.25, 1.5)$);

            \coordinate (R1) at (3, 5);
            %470 Ohms
            \draw[very thick, black!50] (R1) -- ($(R1) + (4, 0)$);
            \draw[fill=Tan ] ($(R1) + (1.4, 0.25)$) rectangle ($(R1) + (2.6, -0.25)$);
            \draw[fill=Yellow] ($(R1) + (1.6, 0.25)$) rectangle ($(R1) + (1.7, -0.25)$);
            \draw[fill=Purple] ($(R1) + (1.8, 0.25)$) rectangle ($(R1) + (1.9, -0.25)$);
            \draw[fill=Brown] ($(R1) + (2.0, 0.25)$) rectangle ($(R1) + (2.1, -0.25)$);
            \draw[fill=Goldenrod] ($(R1) + (2.5, 0.25)$) rectangle ($(R1) + (2.4, -0.25)$);

            \coordinate (R1) at (3, -5);
            %330 Ohms
            \draw[very thick, black!50] (R1) -- ($(R1) + (4, 0)$);
            \draw[fill=Tan ] ($(R1) + (1.4, 0.25)$) rectangle ($(R1) + (2.6, -0.25)$);
            \draw[fill=Orange] ($(R1) + (1.6, 0.25)$) rectangle ($(R1) + (1.7, -0.25)$);
            \draw[fill=Orange] ($(R1) + (1.8, 0.25)$) rectangle ($(R1) + (1.9, -0.25)$);
            \draw[fill=Brown] ($(R1) + (2.0, 0.25)$) rectangle ($(R1) + (2.1, -0.25)$);
            \draw[fill=Goldenrod] ($(R1) + (2.5, 0.25)$) rectangle ($(R1) + (2.4, -0.25)$);

        \end{tikzpicture}
    \end{center}

    % parallel circuit for resistors + multimeter
    \subsubsection*{Circuit \#9: Resistors in Parallel}
    \begin{center}
        \begin{tikzpicture}
            \draw[line width=6mm,YellowOrange] (0, 0) -- (0, 5) -- (10, 5) -- (10, 1);
            \draw[line width=6mm,YellowOrange] (10, -1) -- (10, -5) -- (0, -5) -- (0, 0);
            \draw[line width=6mm,YellowOrange] (10, 5) -- (15, 5) -- (15, 1);

            \draw[line width=6mm,YellowOrange] (5, 5) -- (5, 1);
            \draw[line width=6mm,YellowOrange] (5, -1) -- (5, -5);
            \draw[line width=6mm,YellowOrange] (15, -1) -- (15, -5) -- (10, -5);

            \fill[black!25] (0, 0) circle (10mm);            
            \draw[line width=6mm, dashed,YellowOrange!50,draw opacity=0.5] (0, -0.25) -- (0, -1);
            \draw[line width=6mm,YellowOrange] (0, 0.25) -- (0, 1);
            \node[align=center] at (0, 0) {CR2032};

            \draw[->,>=triangle 45,very thick,red] (1.4, 2) node[right] {A} -- (0.4, 2);
            \draw[->,>=triangle 45,very thick,black] (1.4, -2) node[right] {A} -- (0.4, -2);

            \draw[->,>=triangle 45,very thick,red] (3.6, 3.5) node[left] {B} --(4.6, 3.5);
            \draw[->,>=triangle 45,very thick,black] (3.6, -3.5) node[left] {B} -- (4.6, -3.5);

            \draw[->,>=triangle 45,very thick,red] (8.6, 3.5) node[left] {C} --(9.6, 3.5);
            \draw[->,>=triangle 45,very thick,black] (8.6, -3.5) node[left] {C} -- (9.6, -3.5);            

            \draw[->,>=triangle 45,very thick,red] (13.6, 3.5) node[left] {D} --(14.6, 3.5);
            \draw[->,>=triangle 45,very thick,black] (13.6, -3.5) node[left] {D} -- (14.6, -3.5);           
            
            \coordinate (R1) at (5, -2);
            %220 Ohms
            \draw[very thick, black!50] (R1) -- ($(R1) + (0, 4)$);
            \draw[fill=Tan] ($(R1) + (0.25, 1.4)$) rectangle ($(R1) + (-0.25, 2.6)$);
            \draw[fill=Red] ($(R1) + (0.25, 2.4)$) rectangle ($(R1) + (-0.25, 2.3)$);
            \draw[fill=Red] ($(R1) + (0.25, 2.2)$) rectangle ($(R1) + (-0.25, 2.1)$);
            \draw[fill=Brown] ($(R1) + (0.25, 2.0)$) rectangle ($(R1) + (-0.25, 1.9)$);
            \draw[fill=Goldenrod] ($(R1) + (0.25, 1.6)$) rectangle ($(R1) + (-0.25, 1.5)$);

            \coordinate (R1) at (10, -2);
            %330 Ohms
            \draw[very thick, black!50] (R1) -- ($(R1) + (0, 4)$);
            \draw[fill=Tan] ($(R1) + (0.25, 1.4)$) rectangle ($(R1) + (-0.25, 2.6)$);
            \draw[fill=Orange] ($(R1) + (0.25, 2.4)$) rectangle ($(R1) + (-0.25, 2.3)$);
            \draw[fill=Orange] ($(R1) + (0.25, 2.2)$) rectangle ($(R1) + (-0.25, 2.1)$);
            \draw[fill=Brown] ($(R1) + (0.25, 2.0)$) rectangle ($(R1) + (-0.25, 1.9)$);
            \draw[fill=Goldenrod] ($(R1) + (0.25, 1.6)$) rectangle ($(R1) + (-0.25, 1.5)$);

            \coordinate (R1) at (15, -2);
            %1k Ohms
            \draw[very thick, black!50] (R1) -- ($(R1) + (0, 4)$);
            \draw[fill=Tan] ($(R1) + (0.25, 1.4)$) rectangle ($(R1) + (-0.25, 2.6)$);
            \draw[fill=Red] ($(R1) + (0.25, 2.4)$) rectangle ($(R1) + (-0.25, 2.3)$);
            \draw[fill=Black] ($(R1) + (0.25, 2.2)$) rectangle ($(R1) + (-0.25, 2.1)$);
            \draw[fill=Red] ($(R1) + (0.25, 2.0)$) rectangle ($(R1) + (-0.25, 1.9)$);
            \draw[fill=Goldenrod] ($(R1) + (0.25, 1.6)$) rectangle ($(R1) + (-0.25, 1.5)$);
        \end{tikzpicture}
    \end{center}   
    \pagebreak
 
    \begin{tabularx}{\boxwidth}{| X | }
        \hline
        \FormativeHeader \\\hline
        \QuestionBox{} \\\hline
    \end{tabularx}

    \section{Reflections}
\end{document}