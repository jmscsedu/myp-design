\documentclass[10pt,oneside]{article}

\usepackage[T1]{fontenc}
\usepackage{fontawesome}

\usepackage[paper=a4paper,margin=2cm,bottom=2.5cm]{geometry}
\usepackage[sfdefault,light,condensed]{roboto}
\usepackage[export]{adjustbox}
\usepackage[usenames,dvipsnames,table]{xcolor}

\usepackage{amsmath,amssymb,array,fancyhdr,graphicx,enumitem,lastpage,multicol,tabularx,textcomp,titlesec}
\usepackage{mathtools}

\setlength\extrarowheight{1pt}
\setlength\parindent{0cm}
\renewcommand\headrule{}
\setlength{\footskip}{1.25cm}

\pagestyle{fancy}

\definecolor{BoxHeaderBG}{RGB}{50, 50, 50}
\definecolor{BoxHeaderText}{RGB}{255, 255, 255}

\newcommand{\BoxHeader}[2]{
    \multicolumn{#1}{| >{\bfseries\footnotesize\cellcolor{BoxHeaderBG}\arraybackslash}l |}{
        \textcolor{BoxHeaderText}{#2}
    }
}

\definecolor{ATLHeaderBG}{RGB}{65, 190, 30}
\definecolor{ATLHeaderText}{RGB}{0, 0, 0}

\definecolor{ATLSkillBG}{RGB}{215, 230, 210}
\definecolor{ATLSkillText}{RGB}{0, 0, 0}

\definecolor{DefinitionBoxHeaderBG}{RGB}{30, 30, 110}
\definecolor{DefinitionBoxHeaderText}{RGB}{255, 255, 255}

\definecolor{FormativeHeaderBG}{RGB}{150, 30, 150}
\definecolor{FormativeHeaderText}{RGB}{255, 255, 255}

\definecolor{GlobalContextHeaderBG}{RGB}{255, 255, 150}
\definecolor{GlobalContextHeaderText}{RGB}{0, 0, 0}

\definecolor{KeyConceptHeaderBG}{RGB}{15, 225, 225}
\definecolor{KeyConceptHeaderText}{RGB}{0, 0, 0}

\definecolor{RelatedConceptHeaderBG}{RGB}{15, 170, 170}
\definecolor{RelatedConceptHeaderText}{RGB}{0, 0, 0}

\definecolor{QuestionHeaderBG}{RGB}{240, 240, 240}
\definecolor{QuestionHeaderText}{RGB}{0, 0, 0}

\definecolor{SolutionHeaderBG}{RGB}{225, 150, 110}
\definecolor{SolutionHeaderText}{RGB}{0, 0 , 0}

\definecolor{SummativeHeaderBG}{RGB}{195, 15, 15}
\definecolor{SummativeHeaderText}{RGB}{255, 255, 255}

\definecolor{WarningHeaderBG}{RGB}{250, 220, 125}
\definecolor{WarningHeaderText}{RGB}{0, 0, 0}

\newcommand{\ATLHeader}[1]{
    \cellcolor{ATLHeaderBG}\textcolor{ATLHeaderText}{
        \bfseries\footnotesize
        ATL SKILL (#1) \hfill \faGears
    }
}

\newcommand{\ATLSkill}[1]{
    \cellcolor{ATLSkillBG}\textcolor{ATLSkillText}{
        \itshape #1
    }
}

\newcommand{\DefinitionBoxHeader}{
    \cellcolor{DefinitionBoxHeaderBG}\textcolor{DefinitionBoxHeaderText}{
        \bfseries\footnotesize
        DEFINITIONS \hfill \faPencil
    }
}

\newcommand{\FormativeHeader}{
    \cellcolor{FormativeHeaderBG}\textcolor{FormativeHeaderText}{
        \bfseries\footnotesize
        FORMATIVE ASSESSMENT \hfill \faComments
    }
}

\newcommand{\GlobalContextHeader}[1]{
    \cellcolor{GlobalContextHeaderBG}\textcolor{GlobalContextHeaderText}{
        \bfseries\footnotesize
        GLOBAL CONTEXT (#1) \hfill \faGlobe
    }
}

\newcommand{\KeyConceptHeader}[1]{
    \cellcolor{KeyConceptHeaderBG}\textcolor{KeyConceptHeaderText}{
        \bfseries\footnotesize
        KEY CONCEPT (#1) \hfill \faKey
    }
}

\newcommand{\RelatedConceptHeader}[1]{
    \cellcolor{RelatedConceptHeaderBG}\textcolor{RelatedConceptHeaderText}{
        \bfseries\footnotesize
        RELATED CONCEPT (#1) \hfill \faLink
    }
}

\newcommand{\SolutionHeader}[1]{
    \cellcolor{SolutionHeaderBG}\textcolor{SolutionHeaderText}{
        \bfseries\footnotesize 
        #1 \hfill \faPaste
    }
}

\newcommand{\SummativeHeader}{
    \cellcolor{SummativeHeaderBG}\textcolor{SummativeHeaderText}{
        \bfseries\footnotesize
        SUMMATIVE ASSESSMENT \hfill \faCheck
    }
}

\newcommand{\WarningHeader}[1]{
    \cellcolor{WarningHeaderBG}\textcolor{WarningHeaderText}{
        \textbf{\footnotesize\faExclamationTriangle} {\normalfont #1} \hfill \textbf{\footnotesize\faExclamationTriangle}
    }
}
\newcounter{QuestionCounter}

\newcommand{\QuestionBox}[1]{
    \stepcounter{QuestionCounter}
    \cellcolor{QuestionHeaderBG}\textcolor{QuestionHeaderText}{
        {\bfseries\scriptsize Q\theQuestionCounter} #1
    }
}

\newcommand{\boxwidth}{\linewidth}

\definecolor{ATLHeaderBG}{RGB}{65, 190, 30}
\definecolor{ATLHeaderText}{RGB}{0, 0, 0}

\definecolor{ATLSkillBG}{RGB}{215, 230, 210}
\definecolor{ATLSkillText}{RGB}{0, 0, 0}

\definecolor{DefinitionBoxHeaderBG}{RGB}{30, 30, 110}
\definecolor{DefinitionBoxHeaderText}{RGB}{255, 255, 255}

\definecolor{FormativeHeaderBG}{RGB}{150, 30, 150}
\definecolor{FormativeHeaderText}{RGB}{255, 255, 255}

\definecolor{GlobalContextHeaderBG}{RGB}{255, 255, 150}
\definecolor{GlobalContextHeaderText}{RGB}{0, 0, 0}

\definecolor{KeyConceptHeaderBG}{RGB}{15, 225, 225}
\definecolor{KeyConceptHeaderText}{RGB}{0, 0, 0}

\definecolor{RelatedConceptHeaderBG}{RGB}{15, 170, 170}
\definecolor{RelatedConceptHeaderText}{RGB}{0, 0, 0}

\definecolor{QuestionHeaderBG}{RGB}{240, 240, 240}
\definecolor{QuestionHeaderText}{RGB}{0, 0, 0}

\definecolor{SolutionHeaderBG}{RGB}{225, 150, 110}
\definecolor{SolutionHeaderText}{RGB}{0, 0 , 0}

\definecolor{SummativeHeaderBG}{RGB}{195, 15, 15}
\definecolor{SummativeHeaderText}{RGB}{255, 255, 255}

\newcommand{\ATLHeader}[1]{
    \cellcolor{ATLHeaderBG}\textcolor{ATLHeaderText}{
        \bfseries\footnotesize
        ATL SKILL (#1) \hfill \faGears
    }
}

\newcommand{\ATLSkill}[1]{
    \cellcolor{ATLSkillBG}\textcolor{ATLSkillText}{
        \itshape #1
    }
}

\newcommand{\DefinitionBoxHeader}{
    \cellcolor{DefinitionBoxHeaderBG}\textcolor{DefinitionBoxHeaderText}{
        \bfseries\footnotesize
        DEFINITIONS \hfill \faPencil
    }
}

\newcommand{\FormativeHeader}{
    \cellcolor{FormativeHeaderBG}\textcolor{FormativeHeaderText}{
        \bfseries\footnotesize
        FORMATIVE ASSESSMENT \hfill \faComments
    }
}

\newcommand{\GlobalContextHeader}[1]{
    \cellcolor{GlobalContextHeaderBG}\textcolor{GlobalContextHeaderText}{
        \bfseries\footnotesize
        GLOBAL CONTEXT (#1) \hfill \faGlobe
    }
}

\newcommand{\KeyConceptHeader}[1]{
    \cellcolor{KeyConceptHeaderBG}\textcolor{KeyConceptHeaderText}{
        \bfseries\footnotesize
        KEY CONCEPT (#1) \hfill \faKey
    }
}

\newcommand{\RelatedConceptHeader}[1]{
    \cellcolor{RelatedConceptHeaderBG}\textcolor{RelatedConceptHeaderText}{
        \bfseries\footnotesize
        RELATED CONCEPT (#1) \hfill \faLink
    }
}

\newcommand{\SolutionHeader}[1]{
    \cellcolor{SolutionHeaderBG}\textcolor{SolutionHeaderText}{
        \bfseries\footnotesize 
        #1 \hfill \faPaste
    }
}

\newcommand{\SummativeHeader}{
    \cellcolor{SummativeHeaderBG}\textcolor{SummativeHeaderText}{
        \bfseries\footnotesize
        SUMMATIVE ASSESSMENT \hfill \faCheck
    }
}

\newcommand{\QuestionBox}[1]{
    \cellcolor{QuestionHeaderBG}\textcolor{QuestionHeaderText}{
        #1
    }
}

\lhead{\footnotesize\texttt{U\UnitNumber: \UnitTitle \\ L\LessonNumber: \LessonTitle}}
\rhead{\footnotesize\ttfamily [DESIGN/\CourseName/U\UnitNumber/L\LessonNumber]\\\ }

\lfoot{\includegraphics[height=2cm,valign=c]{Files/logo}}
\cfoot{\footnotesize DESIGN/\CourseName/U\UnitNumber/L\LessonNumber\ | \LessonTitle \\ Woodstock School | Mussoorie, Uttarakhand, India}
\rfoot{\includegraphics[height=2cm,valign=c]{Files/ib-world-school-logo-1-colour}}

\titleformat{\section}{\normalfont\Large\bfseries}{}{0em}{}[{\titlerule[0.5pt]}]
\titleformat{\subsection}{\normalfont\large\bfseries}{}{0em}{}


\usepackage{circuitikz,tikzsymbols}
\usetikzlibrary{arrows}

\def\CourseName{MYP3}

\def\LessonNumber{04}
\def\LessonTitle{Variable Resistance}

\def\UnitNumber{01}
\def\UnitTitle{Circuits \& Electronics}

\begin{document}
    \begin{center}
        \huge\bfseries
        \LessonTitle
    \end{center}

    \section{Before You Begin}
        \begin{tabularx}{\boxwidth}{| X |}
            \hline
            \KeyConceptHeader{Development}\\\hline
            \QuestionBox{The IB defines \emph{development} as ``the act or process of growth, progress or evolution, sometimes through iterative improvements.'' In your own words, describe how the \emph{design cycle} encourages this type of development.}\\\hline
            \ \\[5cm]\hline
        \end{tabularx}

        \medskip
        \begin{tabularx}{\boxwidth}{| X |}
            \hline
            \GlobalContextHeader{Orientation in Space \& Time}\\\hline
            \QuestionBox{What \emph{development} have you see in your life and in what context? You can consider personal, local, regional, or even global developments.}\\\hline
            \ \\[5cm]\hline
        \end{tabularx}

        \medskip
        \begin{tabularx}{\boxwidth}{| X |}
            \hline
            \RelatedConceptHeader{Invention}\\\hline
            \QuestionBox{The IB defines \emph{invention} as ``an entirely novel product or a feature of a product that is unique.'' Using the two definitions given here for \emph{development} and \emph{invention}, describe how development can be furthered through invention.}\\\hline
            \ \\[5cm]\hline
        \end{tabularx}
    \pagebreak

    \section{Technical Background}
    \subsection{Variable Resistors}
    So far, we have been using \emph{fixed-value} resistors in our circuits; however, there are a large variety of devices and applications which require \emph{variable-value} resistors. These resistors will impact the flow of electricity difference based on a numer of environmental or user-controlled factors.

    \medskip
    Although there are a wide variety of these types of resistors, our efforst will focus on only two: the potentiometer and the photoresistor, or light-dependent resistor (LDR).
    
    \bigskip
    \begin{tabularx}{\boxwidth}{| >{\bfseries}p{0.15\boxwidth} | X | >{\centering\arraybackslash}p{0.15\boxwidth} | >{\centering\arraybackslash}p{0.15\boxwidth}| }
        \hline
        \BoxHeader{1}{Name} & \BoxHeader{1}{Description} & \BoxHeader{1}{Symbol} & \BoxHeader{1}{Example} \\\hline
        Photoresistor (LDR) & A photoresistor, or light-dependent resistor (LDR), has a resistance value that varies based on the presence of ambient light. & \raisebox{-0.5cm}{\tikz \draw (0, 0) to [photoresistor] (2, 0);} & ... \\\hline
        Potentiometer & A potentiometer is used to add the ability to vary either voltage or resistance to a circuit through some physical manipulation, often the turning of a knob or position of a slider. & \raisebox{-0.75cm}{\tikz \draw (0, 0) to [european potentiometer] (2, 0);} & ...\\\hline
    \end{tabularx}

    \subsubsection*{The Potentiometer}
    The potentiometer is an interesting device in how simple the classic version actually is. Below is a diagram labeling the essential components of a potentiometer.

    \begin{center}
        \begin{tikzpicture}
            \draw[line width=6mm,Brown!50] (240:1cm) arc (240:-60:1cm);
            \draw[line width=3mm,black!35] (-1, 1) -- (0, 0) -- (0, -1.1);
            \draw (0, -1.1) -- (0, -1.5) node [below] {W};
            \draw (-0.5, -0.9) -- (-0.5, -1.5) node[below] {A};
            \draw (0.5, -0.9) -- (0.5, -1.5) node[below] {B};

            \draw[->,>=latex,very thick,blue] (2, 0) node[right] {resistant material} -- (1.5, 0);
            \draw[->,>=latex,very thick,blue] (1.85, 0.5) node[right,xshift=5,yshift=-2] {wiper} to [out=-20,in=45] (-0.2, 0.5);
            \draw[->,>=latex,very thick,blue] (2, -0.5) node[right] {terminals} to [out=240,in=90] (0, -1);
        \end{tikzpicture}
    \end{center}

    Turning the knokb of the potentiometer changes the position of the ``wiper'' along the resistant material. Depending on the configuration of the circuit, this will either vary source voltage to the rest of the circuit, or introduce variable resistance to the circuit.

    \bigskip
    \begin{minipage}{0.475\boxwidth}
        {\bfseries Varying Voltage}

        When the wiper and one of the other terminals is given a positive charge as in the simple circuit below, then the potentiometer can be thought of as varying the voltage to the circuit.

            \begin{center}
                \begin{circuitikz}
                    \draw (1, 1.5) to [european potentiometer,n=p] (3, 1.5);
                    \draw (0, 0) to [battery1,invert] (0, 1) |- (p.b);
                    \draw (0, 1) |- (p.wiper);
                    \draw (3, 1.5) |- (3, -0.5) to [full led] (1, -0.5) -| (0, 0);
                \end{circuitikz}
            \end{center}
    \end{minipage}
    \begin{minipage}{0.0425\boxwidth}
        \

    \end{minipage}
    \begin{minipage}{0.475\boxwidth}
        {\bfseries Varying Resistance}

        When power is supplied to the wiper and a component is attached to one of the terminals, then the potentiometer can be thought of as varying the resistance to the circuit powering that component.
        
        \begin{center}
            \begin{circuitikz}
                \draw (1, 1.5) to [european potentiometer,n=p] (3, 1.5);
                \draw (0, 0) to [battery1, invert] (0, 1) |- (p.wiper);
                \draw (3, 1.5) |- (3, -0.5) to [full led] (1, -0.5) -| (0, 0);
            \end{circuitikz}
        \end{center}
    \end{minipage}

    \medskip
    \textbf{Note:} The two circuits above are very similar in both form and function. The actual difference between the behaviours of the circuits and why you'd choose one over the other is beyond the scope of this course.

    \pagebreak

    \section{Developing Technical Skills}
    \subsection{Circuit \#11}
    Our first circuit this lesson will use the photoresistor to provide variable resistance for the circuit.

    \subsubsection*{You Will Need}
    \begin{itemize}[noitemsep]
        \item[(1)] CR2032 Battery
        \item[(1)] Photoresistor/LDR
        \item[(1)] LED
        \item[(1)] Roll of Copper Tape
        \item[(1)] Roll of Cellophane Tape   
    \end{itemize}

    \subsubsection*{Directions}
    Create the following paper circuit, then wave your hand over the LDR to observe the effects of variable resistance on the LED.

    \bigskip\medskip
    \begin{center}
        \begin{tikzpicture}
            \draw[line width=6mm,YellowOrange] (0, 1) |- (4, 5)
                                                (6, 5) -| (10, 1)
                                                (10,-1) |- (0, -5) -- (0, 0);
                                                ;

            \begin{scope}[xshift=5cm,yshift=5cm]   % Photoresistor / LDR
                \draw[black!50,very thick] (-2, 0) -- (2, 0);
                \clip (-0.25, -0.23) rectangle (0.25, 0.23);
                \fill[fill=Orange!75] (0, 0) circle (0.25);
                \clip (-0.2, -0.175) rectangle (0.2, 0.175);
                \fill[fill=black!15] (0, 0) circle (0.2);
                \draw[Orange] (0, 0.175) -- (0, 0.1) -- (-0.125, 0.1) -- (-0.125, 0.05) -- (0.125, 0.05) -- (0.125, 0) -- (-0.125, 0) -- (-0.125, -0.05) -- (0.125, -0.05) -- (0.125, -0.1) -- (0, -0.1) -- (0, -0.175);
            \end{scope}

            \fill[black!25] (0, 0) circle (10mm);            
            \draw[line width=6mm, dashed,YellowOrange!50,draw opacity=0.5] (0, -0.25) -- (0, -1);
            \draw[line width=6mm,YellowOrange] (0, 0.25) -- (0, 1);
            \node[align=center] at (0, 0) {CR2032};

            \draw[very thick,black!50] (10, 2) -- (10, -2);
            \fill[left color=red, right color=black!70] ([shift=(-60:2.5mm)]10,0) arc (-60:240:2.5mm);


            \draw[->,>=triangle 45,very thick,red] (2.5, 3.4) node[below] {A} -- (2.5, 4.4);
            \draw[->,>=triangle 45,very thick,black] (7.5, 3.4) node[below] {A} -- (7.5, 4.4);
        \end{tikzpicture}
    \end{center}

    \bigskip
    \begin{tabularx}{\boxwidth}{| X |}
        \hline
        \ATLHeader{Communication Skills}\\\hline
        \ATLSkill{...make inferences and draw conclusions...}\\\hline
        \QuestionBox{Use a multimeter to measure the resistance of the LDR in the above circuit as the light entering it changes. Do the results agree with your idea of how this variable resistor works? Why or why not?}\\\hline
        \ \\[3cm]\hline
    \end{tabularx}
    
    \pagebreak
    \subsection{Circuit \#12}
    The following circuit will allow you to explore the behaviours of the potentiometer, particularly as you turn it in each direction.

    \subsubsection*{You Will Need}
    \begin{itemize}[noitemsep]
        \item[(1)] CR2032 Battery
        \item[(1)] Potentiometer
        \item[(2)] LEDs
        \item[(1)] Roll of Copper Tape
        \item[(1)] Roll of Cellophane Tape    
    \end{itemize}

    \subsubsection*{Directions}
    Create the following paper circuit and manipulate the potentiometer to observe its effects on the two LEDs.

    \medskip
    \textbf{Note:} The gaps between the terminals of the potentiometer are necessary. You may need to trim your copper tape slightly if you are having trouble leaving enough space.

    \bigskip
    \begin{center}
        \begin{tikzpicture}
            \draw[line width=6mm,YellowOrange] (0, 1) |- (8, 5) -- (8, 0);
            \draw[line width=6mm,YellowOrange] (0, -1) |- (12, -5) -- (12, -3);
            \draw[line width=6mm,YellowOrange] (4, -5) -- (4, -3);
            \draw[line width=6mm,YellowOrange] (7.6, 1.3) -| (4, -1);
            \draw[line width=6mm,YellowOrange] (8.4, 1.3) -| (12, -1);

            \fill[black!25] (0, 0) circle (10mm);            
            \draw[line width=6mm, dashed,YellowOrange!50,draw opacity=0.5] (0, -0.25) -- (0, -1);
            \draw[line width=6mm,YellowOrange] (0, 0.25) -- (0, 1);
            \node[align=center] at (0, 0) {CR2032};

            \draw[->,>=triangle 45,very thick,blue] (6, 2) to [out=-0,in=90] (8.35, 1.5);
            \draw[->,>=triangle 45,very thick,blue] (6, 2) node[left] {leave gaps!} to [out=-0,in=90] (7.65, 1.5);
            \begin{scope}[xshift=8cm,rotate=180]
                \draw[fill=brown] (-0.85, 0) rectangle (0.85, -1.05);
                \draw[fill=black!25] (-0.15, -0.9) rectangle (0.15, -1.3);
                \draw[fill=black!25] (-0.4, -0.9) rectangle (-0.7, -1.3);
                \draw[fill=black!25] (0.4, -0.9) rectangle (0.7, -1.3);
                \draw[fill=black!25] (0, 0) circle (0.85);
                \draw[fill=black!15] (0, 0) circle (0.3);
                \fill[black!50] (-0.3, -0.05) rectangle (0.3, 0.05);
            \end{scope}

            \draw[very thick,black!50] (4, 0) -- (4, -4);
            \fill[left color=red, right color=black!70] ([shift=(-60:2.5mm)]4,-2) arc (-60:240:2.5mm);

            \draw[very thick,black!50] (12, 0) -- (12, -4);
            \fill[left color=red, right color=black!70] ([shift=(-60:2.5mm)]12,-2) arc (-60:240:2.5mm);

            \draw[->,>=triangle 45,very thick,red] (1.4, 2) node[right] {A} -- (0.4, 2);
            \draw[->,>=triangle 45,very thick,black] (1.4, -2) node[right] {A} -- (0.4, -2);

            \draw[->,>=triangle 45,very thick,red] 
                (6.6, 3) node[left] {B} -- (7.6, 3);
            \draw[->,>=triangle 45,very thick,red]    
                (9.4, 3) node[right] {C} -- (8.4, 3);
            \draw[->,>=triangle 45,very thick,black]
                (6, -0.2) node[below] {B} -- (6, 0.8);
            \draw[->,>=triangle 45,very thick,black]
                (10, -0.2) node[below] {C} -- (10, 0.8);    
        \end{tikzpicture}
    \end{center}

    \bigskip
    \begin{tabularx}{\boxwidth}{| X |}
        \hline
        \ATLHeader{Communication Skills}\\\hline
        \ATLSkill{...make inferences and draw conclusions...}\\\hline
        \QuestionBox{Using a multimeter to take the appropriate measurements, describe the behaviour of the potentiometer.}\\\hline
        \ \\[4cm]\hline
    \end{tabularx}
    \pagebreak
    
    \begin{tabularx}{\boxwidth}{| X |}
        \hline
        \FormativeHeader\\\hline
        \QuestionBox{Sketch a circuit diagram for a circuit where two potentiometers each manipulate their own LED.}\\\hline
        \ \\[6cm]\hline
        \QuestionBox{Create the paper circuit of your above diagram below.}\\\hline
        \ \\[12cm]\hline
        \QuestionBox{Did you create a parallel or series circuit with your potentiometers and LEDs? Explain your choice.}\\\hline
        \ \\[3cm]\hline
    \end{tabularx}

    \pagebreak
    \section{Reflections}
    \begin{tabularx}{\boxwidth}{| X |}
        \hline
        \ATLHeader{Communication Skills}\\\hline
        \ATLSkill{...make inferences and draw conclusions...}\\\hline
        \QuestionBox{Does your ability to verify your inferences or conclusions through the use of a tool, such as the multimeter, embolden or hinder your guesses? Briefly explain why.}\\\hline
        \ \\[4cm]\hline
        \ATLSkill{...use and interpret a range of discipline-specific terms and symbols...}\\\hline
        \QuestionBox{The paper circuits we have been building use their own symbols. Do these symbols make it easier or harder to understand a circuit as compared to the circuit diagrams we have also been learning? Explain why.}\\\hline
        \ \\[4cm]\hline
    \end{tabularx}

    \bigskip
    \begin{tabularx}{\boxwidth}{| X |}
        \hline
        \QuestionBox{What aspect of this lesson was the most challenging for you? How did you overcome that challenge?}\\\hline
        \ \\[3.5cm]\hline
        \QuestionBox{Select the option which best reflects how confident you are in applying what you have learend in this lesson.}\\\hline
        \, \hfill \Sadey[5][orange] \hfill \Neutrey[5][gray] \hfill \Smiley[5][cyan] \hfill \,\\\hline
        \QuestionBox{What additional questions do you still have about this lesson's content?}\\\hline
        \ \\[3.5cm]\hline
    \end{tabularx}   

\end{document}